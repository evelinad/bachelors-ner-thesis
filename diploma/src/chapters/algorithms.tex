\chapter{Tehnici și Algoritmi Folosiți}

Abilitatea de a reunoaște entități pe care sistemul nu le-a văzut înainte este esențială în preformanța unui sistem NER. O astfel de abilitate se bazează pe recunoașterea și clasificarea de reguli declanșate de diferite feature-uri asocitate atât cu exemple pozitive cât și cu exemple negative. Dacă la început, sistemele NER erau bazate pe reguli acum majoritatea folosesc tehnici avansate de Machine Learning supervizat. 

În multe probleme din NLP, datele vin într-o secvență de caractere, cuvinte, expresii, linii sau propoziții. Putem să gândim problema ca atribuind o clasă pentru fiecare astfel de element din secvență.

Acest lucru este ilustrat în \labelindexref{Figura}{img:sequence-labeling-applications} de mai jos (Jurafsky și Manning, Stanford NLP course) \footnote{\url{http://www.stanford.edu/class/cs124}}:

\fig[scale=0.3]{src/img/sequence-labeling-applications.png}{img:sequence-labeling-applications}{Aplicații ale Sequence Labeling}

Există mai multe 

