% This file contains the abstract of the thesis

Recunoașterea Entităților cu Nume (eng. NER, Named Entity Recognition) are ca scop extragerea și clasificarea dintr-un text a persoanelor, organizațiilor, teritoriilor ș.a. care se identifică printr-un nume propriu.

NER face parte dintr-o subramură a Extragerii Informației (eng. Information Extraction), care are ca scop extragerea de informații dintr-un text care nu are structură (adică nu este adnotat), așa cum este limbajul natural.

Odată cu emergența puternică din ultimii 10 ani a Internetului, cantitatea de informație disponibilă a devenit copleșitoare. Astfel, ceea ce la început a pornit din dorința identificării amenințărilor - sprijinit de Departamentul de Apărare American, Procesarea Limbajului Natural a devenit de actualitate mai mult ca oricând. În contextul acesta, extragerea informației din documente fără structură - limbaj natural - este de o importanța majoră. Înțelegerea acestei informații și procesarea ei este obiectivul principal. În cadrul acestui obiectiv, o subsarcină este furnizarea de răspunsuri la următoarele întrebari: Cine? Ce? Unde?. Aceste răspunsuri se găsesc frecvent în ceea ce numim Entități cu Nume (Named Entities). Ele sunt esențiale pentru a putea înțelege textul la un nivel mai profund. Astfel, în acest document vom studia Identificarea Entităților cu Nume (Named Entity Recognition - NER).
